\documentclass[letterpaper]{scrartcl}
\usepackage[nomarkers,figuresonly]{endfloat}
\renewcommand{\efloatseparator}{\mbox{}}
\usepackage[margin=2cm]{geometry}
\usepackage{graphicx}
\usepackage{amsmath}
\usepackage{hyperref}

\title{Mini Places Challenge}
\subtitle{6.869 Final Project Proposal}
\author{David Bau, Jon Gjengset}

\begin{document}
\maketitle

Our goal is to reach at least a 75\% top-5 accuracy for the Mini Places
Challenge by constructing a Convolutional Neural Net.\ In order to reach
this level of accuracy, we want to explore the following ideas:
\begin{enumerate}
\item Experiment with variations on the number, size and type of layers in our
	network architecture, for example, experiment with more aggressive
	pooling or alternatives to fully connected layers, to try to speed
	training for deeper networks.
\item Experiment with CNN visualization to aid us in identifying
	areas of the network that are not being trained well. See
	\url{http://arxiv.org/abs/1311.2901}.
\item Apply unconventional preprocessing to the training data. For
	example, can a neural network do recognition on FFT input?
\item Apply autoencoders during pre-training. Can we build more
	effective, deeper network using autoencoders? See
	\url{http://www.jmlr.org/papers/volume11/vincent10a/vincent10a.pdf}
	and \url{http://arxiv.org/pdf/1310.8499.pdf}.
\item Explore more traditional pre-processing ideas. In addition to
	blurring, rotation, cropping, and flipping, we could experiment
	with warping the image or perspective distortion.
\item Experiment with `attention''-seeking neural nets that combine a
	recurrent network with a CNN to see if we can focus the CNN's
	attention to interesting parts of the image. See
	\url{http://papers.nips.cc/paper/5542-recurrent-models-of-visual-attention.pdf}.
\end{enumerate}

\section*{Implementation Strategy}

We are set up to work with Theano with Lasagne, although Google just released
TensorFlow, and we are considering using that platform.

Our approach will be to try the above ideas iteratively, keeping the ones
that work and leaving behind the ones that don't.

\end{document}
